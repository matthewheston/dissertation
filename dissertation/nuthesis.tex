%%%%%%%%%%%%%%%%%%%%%%%%%%%%%%%%%%%%%%%%%%%%%%%%%%%%%%%%%%%%%%%%%%%%%%
% nuthesis-template.tex - Miguel A, Lerma - 3/31/2013
%                         mlerma@math.northwestern.edu
%%%%%%%%%%%%%%%%%%%%%%%%%%%%%%%%%%%%%%%%%%%%%%%%%%%%%%%%%%%%%%%%%%%%%%

%%%%%%%%%%%%%%%%%%%%%%%% DISCLAIMER %%%%%%%%%%%%%%%%%%%%%%%%%%%%%%%%%%
% 
% In spite of the effort to accommodate this template and the nuthesis
% class to the official requirements of the university to write a 
% Ph.D. dissertation, it is not possible to guarantee that it will 
% always work, and the author of the dissertation remains responsible
% for checking that such requirements are actually fulfilled by 
% his/her final work.
%
%%%%%%%%%%%%%%%%%%%%%%%%%%%%%%%%%%%%%%%%%%%%%%%%%%%%%%%%%%%%%%%%%%%%%%


\documentclass[12pt]{nuthesis}	% The nuthesis class is based on 
				% amsbook.cls.
				
\usepackage[english]{babel}
\usepackage[utf8x]{inputenc}
\usepackage[T1]{fontenc}
\usepackage[natbibapa]{apacite}
\usepackage{comment}


%%%%%%%%%%%%%%%%%%%%%%%%%%%%%%%%%%%
% DATA OF AUTHOR AND DISSERTATION %
%%%%%%%%%%%%%%%%%%%%%%%%%%%%%%%%%%%

\author{Matthew Heston}

\title{The Effect of Contextual and Relational Variables on Predicting Mobile Responsiveness}

%\degree{DOCTOR OF PHILOSOPHY}  % Default: DOCTOR OF PHILOSOPHY

\field{Technology and Social Behavior}            % Default: Mathematics

%\graduationmonth{June}         % The default is June or December
                                % depending on current date.

%\graduationyear{2003}          % Default: current year.


				% Use \includeonly to select the 
%\includeonly{chap1,chap2,...}	% chapters to include if you are 
				% using the \include command below.
				% This way you can latex only a the 
				% part you are working on, which 
				% is faster than latexing the entire 
				% thesis. 


\begin{document}
%	
%	THE BODY OF YOUR THESIS STARTS HERE
%

%%%%%%%%%%%%%%%%%%%%%%
% Some initial stuff %
%%%%%%%%%%%%%%%%%%%%%%

\frontmatter		% Preliminary pages start here.

\maketitle		% Produces the title page.

\copyrightpage		% Creates the copyright page.


\abstract		% Abstract.

This is the abstract.

\acknowledgements	% Acknowledgements (optional).

Text for acknowledgments.

\preface		% Preface (optional).

This is the preface.


%% A few more optional pages (uncomment if needed)
%
%\listofabbreviations 
%
%This is the list of abbreviations (optional).
%
%\glossary
%
%This is the glossary (optional).
%
%\nomenclature
%
%This is the nomenclature (optional).
%
%% Note that the dedication text must be passed as an argument
%% of the \dedication command
%\dedication{This is the dedication (optional).}
%

\clearpage\phantomsection % needed for the hyperlinks to work correctly
\tableofcontents	% Table of Contents will be automatically
			% generated and placed here.

\clearpage\phantomsection % needed for the hyperlinks to work correctly
\listoftables		% List of Tables and List of Figures will be placed

\clearpage\phantomsection % needed for the hyperlinks to work correctly
\listoffigures		% here, if applicable (optional).



\mainmatter             % Actual text starts here.

%%%%%%%%%%%%%%%%%%%%%%%%%%%
% Actual text starts here %
%%%%%%%%%%%%%%%%%%%%%%%%%%%

% If there is an introduction it must be the first chapter

\chapter{Introduction}

~\citet{walther1992interpersonal} predicted that participants in text-based interaction such as email and instant messaging would find ways to adapt to these media, encoding and decoding relational cues in text and deriving psychological knowledge about one another. Although this broke from many other theories of computer-mediated communication (CMC) at the time which posited that the lack of various nonverbal cues in CMC meant such media were unsuitable for nuanced interpersonal communication and would only be used for unambiguous messages in organizations~\citep[e.g.][]{daft1986organizational,sproull1986reducing}, it is clear now that ~\citeauthor{walther1992interpersonal} was correct in his predictions. In addition to forming friendships with others they meet online~\citep{grinter2006chatting}, users also use CMC to maintain existing relationships~\citep{hu2004friendships}, and often develop more intimate relationships with those they communicate with in CMC due to increased self-disclosure~\citep{valkenburg2009effects}. Drawing on Walther's initial model, work since has focused on how users of these platforms encode relational information in text~\citep[e.g.][]{hancock2007expressing,pirzadeh2014you}.

Text-based interaction has become even more ubiquitous with the widespread adoption of mobile devices. A 2015 Pew Report found that 92\% of American adults own a cell phone, with 68\% of U.S. adults having a smart phone~\citep{anderson2015technology}. While smart phone users report using their device for a variety of reasons, such as following the news and finding local events, text messaging is the most widely used feature~\citep{smith2015us}, with other mobile messaging platforms such as WhatsApp and Kik becoming increasingly popular as well~\citep{duggan2015mobile}. Phone owners use text messaging to communicate with a variety of different types of contacts (e.g., friends, family, and co-workers) for a variety of different reasons, ranging from information seeking to simply ``killing time'' ~\citep{battestini2010large}. Many of the same used to signal emotional and relational information in instant messaging and email, such as emoticons and lexical choice, are used the same way in mobile messaging~\citep{lo2008nonverbal,tossell2012longitudinal}.

One cue that may be different in the mobile context is responsiveness, or the amount of time it takes to respond to a message. It has been suggested that responsiveness serves as a cue in CMC, with quick responses signaling immediacy, care, and presence~\citep{kalman2006pauses,walther1995nonverbal}. Indeed, responsiveness has been demonstrated to affect impressions. ~\citet{cramton2002attribution} discusses how long delays in email response are often misattributed, such that those waiting for a response view the slow responders as intentionally ignoring them. ~\citet{heston2017worth} found that short delays in instant messaging can cause a decrease in social attraction among individuals who know each other.

What makes responsiveness unique in mobile messaging is that users carry their devices with them nearly constantly. We know that this has created the expectation that users are constantly available and has increased expectations for immediate responses~\citep{church2013s}. We know less, however, about what actually affects how quickly users decide to respond in mobile messaging. It is possible, for example, that because they are aware of these expectations for fast responsiveness, users simply respond immediately whenever they are available to do so. There is evidence, however, that users consider their relationship with the person trying to reach them when deciding whether or not to respond~\citep{wohn2015ambient}. The content of a message may also matter, e.g., such that messages deemed more important by a recipient are more likely to get a quick response~\citep{dabbish2005understanding}.

It is likely that all of these factors -- a user's availability, their relationship with the person contacting them, and the content of the message -- affect responsiveness behavior, but no work to date has quantified the relative magnitudes of these different effects. Doing so has important theoretical consequences. If the primary driver of responsiveness is simply how available a user is (i.e., users always respond quickly so long as they are not busy), then we should dismiss the hypothesis that responsiveness is signal of intimacy in CMC, as suggested by ~\citet{kalman2006pauses} and ~\citet{walther1995nonverbal}. If, on the other hand, relational variables, such as how close the message recipient is to the message sender, have a large effect on response time, then responsiveness may be seen as carrying relational information in the same ways that other cues in CMC do.

Understanding what affects mobile responsiveness also has practical implications for the design of mobile text-based interaction platforms. HCI scholars have shown an interest in better supporting affective communication in mobile messaging platforms~\citep[e.g.][]{amin2005sensems}. With regards to responsiveness, recent work has suggested the design awareness displays, showing a message sender the predicted likelihood of getting a response from the person they are trying to contact, thus mitigating negative consequences associated with the expectations for a quick response~\citep{pielot2014didn}. If a user's availability is the main factor affecting whether or not they will respond, such a system might be reasonable. If, however, responsiveness is affected primarily by message content and sender-receiver relationship, such a system might not only be inaccurate, but also lack social nuance and cause other issues.

The goal of this dissertation is to quantify the effects of availability, message content, and sender-receiver relationship on mobile responsiveness. 

\chapter{Theoretical and Conceptual Background}

\chapter{The Current Project}

\chapter{Methods}


\chapter{Discussion}

 \renewcommand\refname{\begin{centering}References\end{centering}}
 \bibliography{references.bib}
 \bibliographystyle{apacite} %or another suitable style.



% \appendix		% Appendix begins here (optional).

%\appendices	        % If more than one appendix chapters,
				% use appendices instead of appendix




\end{document}

