%%%%%%%%%%%%%%%%%%%%%%%%%%%%%%%%%%%%%%%%%%%%%%%%%%%%%%%%%%%%%%%%%%%%%%
% nuthesis-template.tex - Miguel A, Lerma - 3/31/2013
%                         mlerma@math.northwestern.edu
%%%%%%%%%%%%%%%%%%%%%%%%%%%%%%%%%%%%%%%%%%%%%%%%%%%%%%%%%%%%%%%%%%%%%%

%%%%%%%%%%%%%%%%%%%%%%%% DISCLAIMER %%%%%%%%%%%%%%%%%%%%%%%%%%%%%%%%%%
% 
% In spite of the effort to accommodate this template and the nuthesis
% class to the official requirements of the university to write a 
% Ph.D. dissertation, it is not possible to guarantee that it will 
% always work, and the author of the dissertation remains responsible
% for checking that such requirements are actually fulfilled by 
% his/her final work.
%
%%%%%%%%%%%%%%%%%%%%%%%%%%%%%%%%%%%%%%%%%%%%%%%%%%%%%%%%%%%%%%%%%%%%%%


\documentclass[12pt]{nuthesis}	% The nuthesis class is based on 
				% amsbook.cls.
				
\usepackage[english]{babel}
\usepackage[utf8x]{inputenc}
\usepackage[T1]{fontenc}
\usepackage[natbibapa]{apacite}
\usepackage{comment}
\usepackage{todonotes}
\usepackage{graphicx}


%%%%%%%%%%%%%%%%%%%%%%%%%%%%%%%%%%%
% DATA OF AUTHOR AND DISSERTATION %
%%%%%%%%%%%%%%%%%%%%%%%%%%%%%%%%%%%

\author{Matthew Heston}

\title{The Effect of Contextual and Relational Variables on Predicting Mobile Responsiveness}

%\degree{DOCTOR OF PHILOSOPHY}  % Default: DOCTOR OF PHILOSOPHY

\field{Technology and Social Behavior}            % Default: Mathematics

%\graduationmonth{June}         % The default is June or December
                                % depending on current date.

%\graduationyear{2003}          % Default: current year.


				% Use \includeonly to select the 
%\includeonly{chap1,chap2,...}	% chapters to include if you are 
				% using the \include command below.
				% This way you can latex only a the 
				% part you are working on, which 
				% is faster than latexing the entire 
				% thesis. 


\begin{document}
%	
%	THE BODY OF YOUR THESIS STARTS HERE
%

%%%%%%%%%%%%%%%%%%%%%%
% Some initial stuff %
%%%%%%%%%%%%%%%%%%%%%%

\frontmatter		% Preliminary pages start here.

\maketitle		% Produces the title page.

\copyrightpage		% Creates the copyright page.


\abstract		% Abstract.

This is the abstract.

\acknowledgements	% Acknowledgements (optional).

Text for acknowledgments.

\preface		% Preface (optional).

This is the preface.


%% A few more optional pages (uncomment if needed)
%
\listofabbreviations
CMC --- Computer Mediated Communication \\
SIP --- Social Information Processing Theory
%
%This is the list of abbreviations (optional).
%
%\glossary
%
%This is the glossary (optional).
%
%\nomenclature
%
%This is the nomenclature (optional).
%
%% Note that the dedication text must be passed as an argument
%% of the \dedication command
%\dedication{This is the dedication (optional).}
%

\clearpage\phantomsection % needed for the hyperlinks to work correctly
\tableofcontents	% Table of Contents will be automatically
			% generated and placed here.

\clearpage\phantomsection % needed for the hyperlinks to work correctly
\listoftables		% List of Tables and List of Figures will be placed

\clearpage\phantomsection % needed for the hyperlinks to work correctly
\listoffigures		% here, if applicable (optional).



\mainmatter             % Actual text starts here.

%%%%%%%%%%%%%%%%%%%%%%%%%%%
% Actual text starts here %
%%%%%%%%%%%%%%%%%%%%%%%%%%%

% If there is an introduction it must be the first chapter

\chapter{Introduction}

~\citet{walther1992interpersonal} predicted that participants in text-based interaction such as email and instant messaging would find ways to adapt to these media, encoding and decoding relational cues in text and deriving psychological knowledge about one another. Although this broke from many other theories of computer-mediated communication (CMC) at the time which posited that the lack of various nonverbal cues in CMC meant such media were unsuitable for nuanced interpersonal communication and would only be used for unambiguous messages in organizations~\citep[e.g.][]{daft1986organizational,sproull1986reducing}, it is clear now that ~\citeauthor{walther1992interpersonal} was correct in his predictions. In addition to forming friendships with others they meet online~\citep{grinter2006chatting}, users also use CMC to maintain existing relationships~\citep{hu2004friendships}, and often develop more intimate relationships with those they communicate with in CMC due to increased self-disclosure~\citep{valkenburg2009effects}. Drawing on Walther's initial model, work since has focused on how users of these platforms encode relational information in text~\citep[e.g.][]{hancock2007expressing,pirzadeh2014you}.

Text-based interaction has become even more ubiquitous with the widespread adoption of mobile devices. A 2015 Pew Report found that 92\% of American adults own a cell phone, with 68\% of U.S. adults having a smart phone~\citep{anderson2015technology}. While smart phone users report using their device for a variety of reasons, such as following the news and finding local events, text messaging is the most widely used feature~\citep{smith2015us}, with other mobile messaging platforms such as WhatsApp and Kik becoming increasingly popular as well~\citep{duggan2015mobile}. Phone owners use text messaging to communicate with a variety of different types of contacts (e.g., friends, family, and co-workers) for a variety of different reasons, ranging from information seeking to simply ``killing time'' ~\citep{battestini2010large}. Many of the same used to signal emotional and relational information in instant messaging and email, such as emoticons and lexical choice, are used the same way in mobile messaging~\citep{lo2008nonverbal,tossell2012longitudinal}.

One cue that may be different in the mobile context is responsiveness, or the amount of time it takes to respond to a message. It has been suggested that responsiveness serves as a cue in CMC, with quick responses signaling immediacy, care, and presence~\citep{kalman2006pauses,walther1995nonverbal}. Indeed, responsiveness has been demonstrated to affect impressions. ~\citet{cramton2002attribution} discusses how long delays in email response are often misattributed, such that those waiting for a response view the slow responders as intentionally ignoring them. ~\citet{heston2017worth} found that short delays in instant messaging can cause a decrease in social attraction among individuals who know each other.

What makes responsiveness unique in mobile messaging is that users carry their devices with them nearly constantly. We know that this has created the expectation that users are constantly available and has increased expectations for immediate responses~\citep{church2013s}. We know less, however, about what actually affects how quickly users decide to respond in mobile messaging. It is possible, for example, that because they are aware of these expectations for fast responsiveness, users simply respond immediately whenever they are available to do so. There is evidence, however, that users consider their relationship with the person trying to reach them when deciding whether or not to respond~\citep{wohn2015ambient}. The content of a message may also matter, e.g., such that messages deemed more important by a recipient are more likely to get a quick response~\citep{dabbish2005understanding}.

It is likely that all of these factors -- a user's availability, their relationship with the person contacting them, and the content of the message -- affect responsiveness behavior, but no work to date has quantified the relative magnitudes of these different effects. Doing so has important theoretical consequences. If the primary driver of responsiveness is simply how available a user is (i.e., users always respond quickly so long as they are not busy), then we should dismiss the hypothesis that responsiveness is signal of intimacy in CMC, as suggested by ~\citet{kalman2006pauses} and ~\citet{walther1995nonverbal}. If, on the other hand, relational variables, such as how close the message recipient is to the message sender, have a large effect on response time, then responsiveness may be seen as carrying relational information in the same ways that other cues in CMC do.

Understanding what affects mobile responsiveness also has practical implications for the design of mobile text-based interaction platforms. HCI scholars have shown an interest in better supporting affective communication in mobile messaging platforms~\citep[e.g.][]{amin2005sensems}. With regards to responsiveness, recent work has suggested the design awareness displays, showing a message sender the predicted likelihood of getting a response from the person they are trying to contact, thus mitigating negative consequences associated with the expectations for a quick response~\citep{pielot2014didn}. If a user's availability is the main factor affecting whether or not they will respond, such a system might be reasonable. If, however, responsiveness is affected primarily by message content and sender-receiver relationship, such a system might not only be inaccurate, but also lack social nuance and cause other issues.

The goal of this dissertation is to quantify the effects of availability, message content, and sender-receiver relationship on mobile responsiveness. 

\chapter{Background}

\section{Cues in CMC}

Nonverbal communication is an essential part of how humans communicate, and includes a variety of elements such as appearance, touch, and proximity~\citep{burgoon2016nonverbal}. Early work on computer-mediated communication focused on the lack of these nonverbal cues in computer-mediated contexts. This work drew on Social Presence Theory, proposed by ~\citet{short1976social}, which classified various media as existing on a continuum of social presence, or the salience of a particular party in an interaction. Face-to-face communication was considered to have high social presence, whereas written communication has the least. ~\citet{burgoon1984relational} conceptualize of social presence as directly related to nonverbal behavior, with the presence of nonverbal behaviors in FtF interactions leading to a greater sense of intimacy.

Given the use of computers in the workplace during this period, early work in CMC focused on how decreased social presence in CMC would affect workplace behavior. For example, ~\citet{kiesler1984social} hypothesized  the lack of cues in CMC results in lack of leadership, since leaders lack the ability  to demonstrate dominance in an online setting. ~\citet{sproull1986reducing} hypothesized  communication  over email was more ``self-absorbed,'' and framed the results from a survey study  within an organization about email in terms of email providing a new social context  with a different set of norms. ~\citet{dubrovsky1991equalization} developed an experimental design to demonstrate how work groups connected over CMC rather  than  FtF interactions resulted  in different hierarchical group structures.

Media scholars developed a similar line of research under the name of Media Richness Theory~\citep{daft1986organizational}, shifting the focus from how various media affected social presence towards how the limitations of media affected communicators choice in selecting media. The theory suggested that platforms such as email could be highly ambiguous, and therefore users would only elect to use email in cases where their message could not be misconstrued. Empirical work drawing on this theory focused on classifying various media in terms of their ``appropriatenessfor different message types~\citep[e.g.][]{rice1993media}. Even work such as ~\citet{panteli2002richness}, which challenged the notion that CMC systems such as email were necessarily less rich than other media, focused primarily on how users decide to use various communication media.

Such theories comprise what ~\citet{walther2002cues} refers to ``cues filtered out'' theories of CMC, so called because of their emphasis on the lack of nonverbal cues driving behavior. His own theory, Social Information Processing~\citep{walther1992interpersonal}, rejects the notion that the lack of certain nonverbal cues restricts communicators' abilities. In what he calls a ``cues filtered in'' approach to CMC, he suggests communicators adapt to CMC, conveying otherwise nonverbal behavior through cues such as the content and style of their message. Whereas ~\citet{daft1986organizational} may suggest email is simply not appropriate for certain types of communication which may be thought of ambiguous, ~\citet{walther1992interpersonal} would instead suggest email users adapt to the medium, encoding various cues to ensure a lack of ambiguity.

Early empirical work testing SIP hypotheses focused on impression formation, comparing experimental decision making groups impressions of one another in FtF and CMC settings. Since then, work in HCI has focused on how users of text-based platforms encode various emotions. In an experimental study, ~\citet{hancock2007expressing} assigned dyads to express various emotions over instant messaging. They found cues such as the use of emoticons or lexical choice was associated with encoding certain emotions, which were readily decoded by conversation partners. ~\citet{derks2008emoticons} studied the use of such cues not just among strangers, but also among friends, finding that paralinguistic cue use such as emoticons varied across relationship types. ~\citet{pirzadeh2014you} studied various the use of various cues such as punctuation and spelling in informal conversation between college friends, finding different cues were used to convey different emotions. ~\citet{hsieh2017playfulness} found that such cues are also associated with ``playfulness'' in mobile messaging. In short, SIP continues to influence work in CMC and in the HCI space, with a focus on understanding how users adapt to text-based communication platforms and how emotional and relational information get encoded.

\subsection{Responsiveness}

Chronemics, or the use of time, can be considered another nonverbal communicative behavior interpreted in interpersonal communication~\citep{burgoon2016nonverbal}. Within CMC studies, one element of chronemic behavior that has received focus is responsiveness, or how long it takes a user to respond to an incoming message~\citep[e.g.][]{heston2017worth,kalman2006pauses,kalman2011online}. With regard to responsiveness in CMC, conceptualizing of conversation as a series of turns is useful insomuch as it provides a framework that suggests participants in CMC rely on cues in deciding on responsiveness behavior, and that breakdowns in conversation occur when these cues are missed.

In FtF interaction, gaps in conversation are also meaningful. ~\citet{mclaughlin1984conversation} details how different short gaps in conversation are a useful element of the turn-taking procedure, where speakers can interpret the gaps as a turn-allocation method. However, when a gap exceeds a certain limit, conversation partners may feel like conversation has broken down. In an experimental study, ~\citet{mclaughlin1982awkward} found that gaps of more than a few seconds led to conversation partners being rated as less competent conversationalists.

Within CMC, a gap of a few seconds may be tolerated. Nevertheless, evidence suggests users of different CMC platforms do have expectations about responsiveness. \citet{cramton2002attribution} found that one issue with geographically dispersed virtual work groups is that time zone differences led to delays in email responsiveness, which in turn led to frustration among team members. ~\citet{kalman2011online} similarly found delays in email responsiveness could lead to negative impressions. ~\citet{heston2017worth} found that even short delays (~10 seconds) in dyadic instant messaging could lead to worse impressions between known acquaintances.

\subsection{Mobile Devices}

In a study of making calls on cell phones, ~\citet{avrahami2007improving} discuss how callers have limited information about the receiver's state, which can lead them to make calls at inappropriate times. This points to a characteristic of communication through mobile devices that distinguishes it from other media, which is that message senders can initiate contact at any time, even without knowledge of the message receiver's state.

A majority of mobile communication now takes place through text-based platforms, especially SMS~\citep{anderson2015technology,battestini2010large,smith2015us}, but also through other mobile messaging applications~\citep{church2013s,duggan2015mobile}. The use of these platforms to stay in nearly constant contact with others has interested communication scholars in how mobile device owners use these devices. ~\citet{ling200210} discussed the use of mobile devices for \textit{micro-coordination} --- the ability to, for example, let someone know you're running a few minutes late while stuck in traffic. In a study of adolescents, ~\citet{kasesniemi200211} found that the ability to stay in contact with others served an important role in how they developed their social relationships. ~\citet{pettegrew2015smart} found that mobile device owners use their devices to facilitate continuous contact with others across a variety of different types of relationships.

At the same time, mobile users may not always wish to be constantly reachable. ~\citet{ames2013managing}, for example, found that some device owners use various strategies, such as turning their phone onto ``airplane mode,'' to avoid being interrupted. In an interview study of young adults, ~\citet{wohn2015ambient} found that mobile device owners consider a variety of factors when deciding whether to respond to an incoming message immediately or delay response until a more convenient time.

\section{Relational Differences in Communication}




\chapter{The Current Project}

Responsiveness may be seen as a cue in the Social Information Processing theory of CMC. CMC scholars have suggested that fast responsiveness may be associated with immediacy and care in CMC \citep{kalman2006pauses,walther1995nonverbal}, and experimental studies have demonstrated that slow responsiveness across different text-based platforms can lead to negative evaluations~\citep{heston2017worth,kalman2011online}.

However, most cues that have been studied from a SIP perspective are explicitly encoded by a sender. Studies such as ~\citet{hancock2007expressing} and ~\citet{pirzadeh2012expression} focus on cues such as word choice, punctuation, and emoticons, all of which a message sender can choose to alter when writing a message. Responsiveness, on the other hand, may not always be a conscious choice, as an individual may not respond simply because he is busy. Responsiveness is affected by many different factors.

The first is likely simply the message receiver's availability. In addition to simply not having access to their device, ~\citet{avrahami2007improving} noted various situational factors that could affect a cell phone owner's decision to take a call, such as if they were currently with a group and would be soon as rude for leaving to accept the call. Even now as social norms shift and being on a phone to respond to a message may seem less inappropriate~\citep{rainie2015americans}, individual users likely have their own set of guidelines that govern a subjective sense of being available for incoming mobile messages.

If availability were the primary factor driving responsiveness behavior, responsiveness would represent a unique type of cue from a SIP perspective: one that is ``encoded'' almost entirely by external forces, but that nevertheless is decoded in a way that affects impressions of a conversation partner in CMC. Investigating the relationship between availability and responsiveness also provides a baseline to compare how important other factors are in affecting responsiveness. My first research question then is:

\textit{RQ1: What is the relationship between a user's current availability and their mobile responsiveness?}

Other factors may also affect responsiveness, however. From a turn-taking perspective of conversation~\citep{sacks1974simplest}, a participant in a conversation will know how to use his turn based on various cues from the previous turn. In mobile messaging, a user might decide to respond quickly based on certain attributes of a received message. For example, someone might be busy and not want to get involved in a text conversation, but nevertheless choose to respond right away to an request for information from someone, under the assumption that that will be the end of their exchange. Furthermore, certain cues in CMC have been associated with heightened urgency or immediacy~\citep{nguyen2014lexical}, which may cause a message receiver to respond more quickly. These effects may ``override'' a user's current availability, assuming they are available enough to see the message.

If these effects were strong, responsiveness more closely resembles the type of cues hypothesized by scholars such as ~\citet{walther1995nonverbal} and ~\citet{kalman2006pauses}, where the responsiveness behavior more closely resembles a deliberate choice made by a conversation participant in response to cues they perceive in a received message. My second research question is:

\textit{RQ2: What is the relationship between attributes of an incoming message, such as perceived urgency, and responsiveness? Is this relationship affected by a user's current availability?}

Finally, the relationship between a message sender and receiver could also affect responsiveness decisions. Work within social pragmatics has suggested various relational attributes that affect the conversation dynamics between parties~\citep{brown1987politeness,goldberg1990interrupting,west1979against,wolfson1990bulge}, and qualitative studies of mobile messaging have suggested social pressure can affect responsiveness behavior~\citep{church2013s}. In particular, the degree closeness between individuals, sometimes referred to as intimacy, along with status differences between individuals have been empirically shown to affect various nonverbal communication behaviors~\citep{guerrero1991waxing,henley1973power,leffler1982effects,sternglanz2004reading}.  At the same time, recent work has argued certain social pragmatic behavior that exists FtF is less important in CMC~\citep{schulze2017knowledge,stromer2015context}, so the relational attributes previously focused on by conversation analysts may not be as important in CMC, at least with regard to responsiveness decisions. In the same way as message-level attributes may ``override'' situational factors, the same may occur with these relational variables, i.e., a user may respond quickly to his boss even if he is busy because he feels pressure to seem available given their relationship.

Quantifying the effect of relational variables on responsiveness behavior will both allow us to see if the same relational attributes demonstrated to affect conversation behavior FtF also exists in CMC, as well as understand how responsiveness as a cue might vary based on relationship, building on work that has studied how other SIP cues change based on relational context~\citep[e.g.,][]{hancock2007expressing}. My final research question is:

\textit{RQ3: How do intimacy and status between a message sender and message receiver affect responsiveness? Is this effect affected by a user's current availability?}


\chapter{Methods}


\chapter{Discussion}

 \renewcommand\refname{\begin{centering}References\end{centering}}
 \bibliography{references.bib}
 \bibliographystyle{apacite} %or another suitable style.



% \appendix		% Appendix begins here (optional).

%\appendices	        % If more than one appendix chapters,
				% use appendices instead of appendix




\end{document}

